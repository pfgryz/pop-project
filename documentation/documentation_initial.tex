\documentclass[titlepage]{article}

\usepackage[english,polish]{babel}
\usepackage[T1]{fontenc}
\usepackage[utf8]{inputenc}

\usepackage{mathtools}
\usepackage{array}
\usepackage{hyperref}
\usepackage[noend]{algpseudocode}
\usepackage[linesnumbered,ruled,vlined]{algorithm2e}

\usepackage[a4paper,top=2cm,bottom=2cm,left=3cm,right=3cm,marginparwidth=1.75cm]{geometry}

\SetKwProg{CustomAlgorithm}{Algorithm}{}{}
\SetKwFunction{EvolutionaryAlgorithm}{EvolutionaryAlgorithm}

\title{
    {
        \huge Dokumentacja wstępna
    } \\
    POP 2023Z
}
\author {
    Anna Schäfer  \\
    Patryk Filip Gryz
}

\begin{document}

    \maketitle

    \newpage

    \tableofcontents

    \newpage

    \section{
        Opis zadania
    }
    W ramach naszego projektu mamy za zadanie pomóc świętemu Mikołajowi zoptymalizować sposób załadunku sań i 
    drogę jaką musi pokonać z prezentami z bieguna północnego, na którym znajdują się prezenty.

    \section{
        Dane
    }
        \subsection{
            Analiza danych
        }
            Dostarczone dane dla zadania składają się z listy prezentów. Prezenty identyfikowane są poprzez:
            \begin{itemize}
                \item identyfikator prezentu
                \item wagę prezentu
                \item długość i szerokość geograficzną
            \end{itemize}    

            Na podstawie danych można oszacować, że ilość sań w pesymistycznym przypadku może wynosić tyle ile jest prezentów.

        \subsection{
            Wybór danych
        }
            Ze względu na ograniczenia sprzętowe nie będzie możliwe przetworzenie pełnego zbioru danych, 
            dlatego dokonaniliśmy podpróbkowania zbioru oryginalnego, w efekcie czego powstały trzy pliki zawierającego dane:
            \begin{itemize}
                \item gifts25000.csv - zawierający 25 tysięcy prezentów (25\% wszystkich danych)
                \item gifts10000.csv - zawierający 10 tysięcy prezentów (10\% wszystkich danych)
                \item gifts1000.csv - zawierający 1 tysiąc prezentów (1\% wszystkich danych)
            \end{itemize} 

            Najmniejszy zbiór zostanie wykorzystany w głównej mierze do badania poprawności implementacji.

    \section{
        Wstępna propozycja rozwiązania
    }
        \subsection{
            Metoda
        }
            Proponowaną wstępną metodą rozwiązania zadania jest wykorzystanie algorytmu ewolucyjnego
            do znalezienia rozwiązania z różnymi implementacjami i parametrami funkcji selekcji, 
            mutacji i sukcesji.

            \begin{algorithm}[H]
                \CustomAlgorithm{\EvolutionaryAlgorithm{
                    GeneratePopulation,
                    Evaluation,
                    Selection,
                    Mutation,
                    Succession,
                    SelectionParameters,
                    MutationParameters,
                    SuccessionParameters,
                    PopulationSize,
                    Iterations
                }} {

                    $P \gets GeneratePopulation(PopulationSize)$

                    $t \gets 0$

                    $H \gets P^0$

                    \While{$t < Iterations$}{
                        \For{
                            $i \in$ $1:PopulationSize$
                        }{
                            $O^{t}_{i} \gets Mutation(Selection(P^t, SelectionParameters), MutationParameters)$
                        }

                        $P^{t} \gets  Succession(O^{t}, SuccessionParameters)$

                        $H \gets H \cup P^{t}$

                        $t \gets t + 1$
                    }
                }
            \end{algorithm}

            Dla algorytmu osobnikiem jest zbiór sań, które przechowują informacje o tym, 
            jakie prezenty w jakich kolejnościach odwiedzają. Będzie więc to macierz trójwymiarowa, 
            gdzie wymiarami są odpowiednio: lista sań, lista prezentów oraz pozycja prezentu w saniach. 
            Z uwagi na ograniczone zasoby sprzętowe osobnik zostanie zaimplementowany jako wektor sań, 
            a każde sanie jako macierz rzadka przechowująca pary postaci $(nr_{prezentu}, kolejnosc)$.

        \subsection{
            Warianty rozwiązania
        }
            \subsubsection{
                Wariant I: Zamiana dowolnych prezentów między dowolnymi saniami
            }
                \begin{itemize}
                    \item Selekcja - turniejowa z wielkością turnieju stałą, równą $2$
                    \item Mutacja - polegająca na wybraniu pary sań i wymianie losowego prezentu z jednych sań i wstawienie na losową pozycję do drugich sań
                    \item Sukcesja - generacyjna z zapamiętaniem najlepszego osiągniętego wyniku
                    \item Parametry: prawdopodobieństwo mutacji
                \end{itemize}

            \subsubsection{
                Wariant II: Zamiana prezentów na ostatnich pozycjach w saniach + zamiana prezentów w saniach
            }
                Metoda będzie składała się z dwóch części: \\

                \noindent W pierwszej nastąpi rozłożenie prezentów między saniami:
                \begin{itemize}
                    \item Selekcja - turniejowa z wielkością turnieju stałą, równą $2$
                    \item Mutacja - polegająca na wybraniu pary sań i transferze ostatniego prezentu z jednych sań do drugich sań
                    \item Sukcesja - generacyjna z zapamiętaniem najlepszego osiągniętego wyniku
                    \item Parametry: prawdopodobieństwo mutacji
                \end{itemize}

                \noindent W drugiej nastąpi optymalizacja ułożenia prezentów w obrębie sań:
                \begin{itemize}
                    \item Selekcja - turniejowa z wielkością turnieju stałą, równą $2$
                    \item Mutacja - polegająca na losowej zamianie prezentów w obrębie wybranych losowo sań
                    \item Sukcesja - generacyjna z zapamiętaniem najlepszego osiągniętego wyniku
                    \item Parametry: prawdopodobieństwo mutacji 
                \end{itemize}

            \subsubsection{
                Generowanie populacji
            }
                Sprawdzone zostanie jaki wpływ na rozwiązanie ma wygenerowana populacja startowa. 
                Poddane eksperymentom zostaną 3 scenariusze losowania populacji dla przedstawionych wariantów rozwiązania:

                \begin{itemize}
                    \item Jednostajne rozłożenie prezentów - prezenty zostaną losowo (z rozkładem jednostajnym) rozłożone między saniami
                    \item Równomierne rozłożenie prezentów - każde sanie będą miały dokładnie jeden prezent
                    \item Gaussowskie rozłożenie prezentów - prezenty zostaną rozłożone między saniami z wykorzystaniem rozkładu normalnego
                \end{itemize}

        \subsection{
            Funkcja celu
        }
            Celem zadania jest minimalizacja sumy ważonych zużyć reniferów $WRW$ \textit{Weighted Reindeer Weariness}
            dla wszystkich sań przypisanych do realizacji zadania transportu prezentów. \\ \\
            Współczynnik $WRW$ dla sań zdefiniowany jest jako suma $dystans * waga$,
            dla wszystkich pokonanych dystansów i przekłada się na matematyczne wyrażenie postaci:
            \[
                WRW_{s} = \sum^{g}_{i=1} [
                    (
                        \sum^{g}_{k=1} w_{k} - 
                        \sum^{i}_{k=1} w_{k} + 10
                    ) \cdot
                    Dist(Loc_{i}, Loc_{i-1})
                ]
            \]
            gdzie:
            \[
                \begin{array}{l}
                    s - \mbox{sanie} \\
                    g - \mbox{ilość prezentów w saniach} \\
                    Dist - \mbox{funkcja dystansu między punktami (Formuła Haversine)} \\
                    Loc_{i} - \mbox{lokalizacja i-tego prezentu z sań} \\
                \end{array}
            \]
            \\

            \noindent Funkcja celu zatem będzie miała postać:
            \[
                q(X) = \sum_{x \in X} WRW_{x}
            \]

    \section{
        Mierzenie jakości rozwiązań
    }
        Zostanie porównane kilka wariantów rozwiązania na podstawie efektywności, niezawodności oraz jakości rozwiązania. 
        Do określania jakości rozwiązania zostaną wykorzystane wykresy:
        \begin{itemize}
            \item krzywej zbieżności - reprezentując wartości funkcji celu wraz z ilością ewaluacji funkcji celu
            \item krzywej ECDF - obrazującej, jak szybko ulega poprawie jakość rozwiązania dla każdego z wariantów
        \end{itemize}

\end{document}